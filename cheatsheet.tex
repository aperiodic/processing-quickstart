%%
%% This is file `leaflet-manual.tex',
%% generated with the docstrip utility.
%%
%% The original source files were:
%%
%% leaflet.dtx  (with options: `manual')
%%
%% Copyright (C) 2003, 2004
%% Rolf Niepraschk, Rolf.Niepraschk@ptb.de
%% Hubert Gaesslein, HubertJG@open.mind.de
%%
%% This work may be distributed and/or modified under the
%% conditions of the LaTeX Project Public License, either version 1.3
%% of this license or (at your option) any later version.
%% The latest version of this license is in
%%   http://www.latex-project.org/lppl.txt
%% and version 1.3 or later is part of all distributions of LaTeX
%% version 2003/12/01 or later.
%%
%% This work has the LPPL maintenance status "author-maintained".
%%
\def\filename{leaflet-manual.tex}
\def\fileversion{v1.0d}   % change this when leaflet-manual changed, too.
\def\filedate{2012/06/04}
\def\docdate {2012/06/04} % change this when leaflet-manual changed, too.
\listfiles
\errorcontextlines=99
\documentclass[
%%notumble,
%%nofoldmark,
%%dvipdfm,
%%portrait,
%%titlepage,
%%nocombine,
%%a3paper,
%%debug,
%%nospecialtricks,
%%draft,
]{leaflet}


\renewcommand*\foldmarkrule{.3mm}
\renewcommand*\foldmarklength{5mm}

\usepackage[T1]{fontenc}
\usepackage{textcomp}
\usepackage{mathptmx}
\usepackage[scaled=0.9]{helvet}
\makeatletter
\def\ptmTeX{T\kern-.1667em\lower.5ex\hbox{E}\kern-.075emX\@}
\DeclareRobustCommand{\ptmLaTeX}{L\kern-.3em
        {\setbox0\hbox{T}%
         %\vb@xt@ % :-)
         \vbox to\ht0{\hbox{%
                            \csname S@\f@size\endcsname
                            \fontsize\sf@size\z@
                            \math@fontsfalse\selectfont
                            A}%
                      \vss}%
        }%
        \kern-.12em
        \ptmTeX}
\makeatother
\let\TeX=\ptmTeX
\let\LaTeX=\ptmLaTeX
\usepackage{shortvrb}
\MakeShortVerb{\|}
\usepackage{url}
\usepackage{graphicx}
\usepackage[dvipsnames,usenames]{color}
\usepackage{listings}
\definecolor{LIGHTGRAY}{gray}{.9}

%%%%\renewcommand{\descfont}{\normalfont}
\newcommand\Lpack[1]{\textsf{#1}}
\newcommand\Lclass[1]{\textsf{#1}}
\newcommand\Lopt[1]{\texttt{#1}}
\newcommand\Lprog[1]{\textit{#1}}

\newcommand*\defaultmarker{\textsuperscript\textasteriskcentered}

\title{Processing Kickstart\vspace{-2ex}}
\author{%
  Dan Lidral-Porter\vspace{-2ex}
}

\CutLine*{1}
\CutLine*{6}

%AddToBackground{5}{%  Background of a small page
% \put(0,0){\textcolor{Cerulean}{\rule{\paperwidth}{\paperheight}}}}
%
%AddToBackground*{2}{% Background of a large page
% \put(\LenToUnit{.5\paperwidth},\LenToUnit{.5\paperheight}){%
%   \makebox(0,0)[c]{%
%     \resizebox{.9\paperwidth}{!}{\rotatebox{35.26}{%
%       \textsf{\textbf{\textcolor{LIGHTGRAY}{BACKGROUND}}}}}}}}

\begin{document}

\definecolor{framegrey}{rgb}{0.7,0.7,0.7}

\lstset{language=Java
       ,basicstyle=\ttfamily
       ,frame=single
       ,rulecolor=\color{framegrey}
       }

\maketitle
\thispagestyle{empty}

%%\LARGE

\tableofcontents

\section{The Canvas}

Processing artworks are referred to as \textit{sketches}.
Every sketch begins with defining the size of the window it will use, called the \textit{canvas}

The \texttt{size} command is used to define the size of the canvas, in pixels.
The general form of the command is \texttt{size(width, height)}.
Both the width and the height must be whole numbers (no decimal points).

Here's an example:

\begin{lstlisting}
size(1280, 720);
\end{lstlisting}

This creates a canvas that is 1280 pixels wide and 720 pixels high.
Try creating a sketch with only a size command, and change the numbers to experiment with the width and height.

A location within the canvas is described using two numbers, called the \textit{coordinates}.
These numbers are:

\begin{enumerate}
  \item How far away the location is from the left side of the canvas.
        This number is called the \textit{x-position}, because the horizontal direction of the canvas is called its \textit{x-axis}.

  \item How far down the canvas the location is.
        This number is called the \textit{y-position}, since the name for the vertical direction of the canvas is the \textit{y-axis}.
\end{enumerate}

We will usually display coordinates as \texttt{(x, y)}, where \texttt{x} is the x-position, and \texttt{y} is the y-position.

As an example, if our canvas is 1000 pixels wide and 500 pixels tall, then the coordinates of its center are \texttt{(500, 250)}.

\textbf{Exercise}: where is the point with coordinates \texttt{(0,0)}?

\section{Color}

\section{Shapes}

\section{All Together}


\begin{thebibliography}{000}
\bibitem{cit:processing-ref}
  \textsc{B. Fry, C. Reas, et. al.}: The Processing Language Reference. 2001-\\
  \url{http://processing.org/reference}
\end{thebibliography}

\loggingall
\end{document}
\endinput
%%
%% End of file `leaflet-manual.tex'.
