%%
%% This is file `leaflet-manual.tex',
%% generated with the docstrip utility.
%%
%% The original source files were:
%%
%% leaflet.dtx  (with options: `manual')
%%
%% Copyright (C) 2003, 2004
%% Rolf Niepraschk, Rolf.Niepraschk@ptb.de
%% Hubert Gaesslein, HubertJG@open.mind.de
%%
%% This work may be distributed and/or modified under the
%% conditions of the LaTeX Project Public License, either version 1.3
%% of this license or (at your option) any later version.
%% The latest version of this license is in
%%   http://www.latex-project.org/lppl.txt
%% and version 1.3 or later is part of all distributions of LaTeX
%% version 2003/12/01 or later.
%%
%% This work has the LPPL maintenance status "author-maintained".
%%
\def\filename{leaflet-manual.tex}
\def\fileversion{v1.0d}   % change this when leaflet-manual changed, too.
\def\filedate{2012/06/04}
\def\docdate {2012/06/04} % change this when leaflet-manual changed, too.
\listfiles
\errorcontextlines=99
\documentclass[
%%notumble,
%%nofoldmark,
%%dvipdfm,
%%portrait,
%%titlepage,
%%nocombine,
%%a3paper,
%%debug,
%%nospecialtricks,
%%draft,
]{leaflet}


\renewcommand*\foldmarkrule{.3mm}
\renewcommand*\foldmarklength{5mm}

\usepackage[T1]{fontenc}
\usepackage{textcomp}
\usepackage{mathptmx}
\usepackage[scaled=0.9]{helvet}
\makeatletter
\def\ptmTeX{T\kern-.1667em\lower.5ex\hbox{E}\kern-.075emX\@}
\DeclareRobustCommand{\ptmLaTeX}{L\kern-.3em
        {\setbox0\hbox{T}%
         %\vb@xt@ % :-)
         \vbox to\ht0{\hbox{%
                            \csname S@\f@size\endcsname
                            \fontsize\sf@size\z@
                            \math@fontsfalse\selectfont
                            A}%
                      \vss}%
        }%
        \kern-.12em
        \ptmTeX}
\makeatother
\let\TeX=\ptmTeX
\let\LaTeX=\ptmLaTeX
\usepackage{shortvrb}
\MakeShortVerb{\|}
\usepackage{url}
\usepackage{graphicx}
\usepackage[dvipsnames,usenames]{color}
\usepackage{listings}
\definecolor{LIGHTGRAY}{gray}{.9}

%%%%\renewcommand{\descfont}{\normalfont}
\newcommand\Lpack[1]{\textsf{#1}}
\newcommand\Lclass[1]{\textsf{#1}}
\newcommand\Lopt[1]{\texttt{#1}}
\newcommand\Lprog[1]{\textit{#1}}

\newcommand*\defaultmarker{\textsuperscript\textasteriskcentered}

\vspace{-2em}

\title{\vspace{-1.5em}Dynamic Processing!\\
Functions \& Animation\vspace{-2ex}}
\author{%
  Dan Lidral-Porter\vspace{-2ex}
}
\date{}

\CutLine*{1}
\CutLine*{6}

%AddToBackground{5}{%  Background of a small page
% \put(0,0){\textcolor{Cerulean}{\rule{\paperwidth}{\paperheight}}}}
%
%AddToBackground*{2}{% Background of a large page
% \put(\LenToUnit{.5\paperwidth},\LenToUnit{.5\paperheight}){%
%   \makebox(0,0)[c]{%
%     \resizebox{.9\paperwidth}{!}{\rotatebox{35.26}{%
%       \textsf{\textbf{\textcolor{LIGHTGRAY}{BACKGROUND}}}}}}}}

\begin{document}

\definecolor{framegrey}{rgb}{0.8,0.8,0.8}

\lstset{language=Java
       ,basicstyle=\ttfamily
       ,frame=single
       ,rulecolor=\color{framegrey}
       }

\maketitle
\thispagestyle{empty}

%%\LARGE

\vspace{-2em}

\section{Writing Functions}

You've been using Processing's built-in functions such as \texttt{size}, \texttt{color}, and \texttt{rect}, but you can write functions of your own.
Functions need to be defined before you can use them.
A \textit{function definition} has this general form:

\begin{lstlisting}
<type> <name>(<arguments>) {
  ...
  return <instance-of-type>;
}
\end{lstlisting}
\vspace{-0.5em}
where
\begin{itemize}
  \item \texttt{<type>} defines what sort of value the function will return.
        It can be a variable type, like \texttt{int} or \texttt{float}, or the special function type \texttt{void}, which means "no value."
  \item \texttt{<name>} is the function's name, which is what you'll type to use the function.
  \item \texttt{<arguments>} can be any number of \textit{argument specifications} (even zero), seperated by commas.
        Each argument specification is exactly the same as a variable definition: a variable type, and then a name.
  \item The \texttt{...} represents any number of statements.
  \item The \texttt{<instance-of-type>} is any value or variable with the same type as the function (that is, it has type \texttt{<type>}).
        If the type is \texttt{void}, then skip this part (because the function doesn't return any value).
\end{itemize}

There's a lot of moving parts here, so I'll show you a concrete example.
The following excerpt defines a function named \texttt{multiply} that has type \texttt{int}, takes two \texttt{int} arguments named \texttt{x} and \texttt{y}, and returns the result of multiplying \texttt{x} and \texttt{y} together.

\begin{lstlisting}
int multiply(int x, int y) {
  return (x * y);
}
\end{lstlisting}

\textbf{Exercise 3.1:} go back to the previous page and compare the general function definition form with the specific definition of \texttt{multiply}.
Do you see how each part of the general form is used in the definition of \texttt{multiply}?

\textbf{Exercise 3.2:} write your own function named \texttt{multiply3} that has type \texttt{float}, takes three \texttt{float} arguments, and returns the result of multiplying all three of its arguments together.

\textbf{Exercise 3.3:} write a function that takes zero arguments.

\subsection{Function Signatures}

Documentation references will often use a function and its arguments together in a form called the \textit{function signature}.
The Processing reference documentation[1] uses function signatures to show all the different ways that a function can be called in the "Syntax" section, and gives individual explanations for all its arguments below in the "Parameters" section (\textit{parameters} is another term for arguments).

The signature is the first part of a function definition, but has all of the types omitted
The signature of the \texttt{multiply} function that was defined previously is \texttt{multiply(x, y)}.

Let me introduce a new function by way of its signature.
The signature is \texttt{divide\_width(slices)}.
You can use \texttt{divide\_width} to divy up the canvas horizontally into however many \texttt{slices} you give it.
It returns the pixel width of $1/\texttt{slices}$ of the canvas.
For example, if \texttt{slices} is 2, it returns half of the canvas's width; if it's 4, it returns a quarter, and so on.

Here's the definiton of \texttt{divide\_width}:
\begin{lstlisting}
int divide_width(int slices) {
  return width / slices;
}
\end{lstlisting}

\textbf{Exercise 3.4} define \texttt{divide\_height}.

Since functions return a value, you can use a function instead of typing literal numeric value in a variable definition.

For example you could define
\begin{lstlisting}
int square_width = divide_width(8);
\end{lstlisting}
\vspace{-0.5em}
to draw squares that are $1/8$ the width of the canvas.

\section{Get on With the Animating Already!}

You might well ask what functions have to do with making things move.
The answer is that Processing animates sketches by looking for a function named \texttt{draw}, and calling it over and over again.

The point of \texttt{draw} is to repeat a bunch of Procesing commands in order to draw things on your canvas.
In this way, \texttt{draw} is different from \texttt{divide\_width} because the point of calling \texttt{draw} is not to get some value back to use.
Its type is \texttt{void}, the special "no value" function type I mentioned on the previous page.
In fact, \texttt{draw} doesn't even take any arguments, either!

\textbf{Exercise 3.5:} see what happens when you try to use \texttt{draw} to define the value of an \texttt{int} variable.

If Processing finds that you have defined a \texttt{draw} function in your sketch, it runs it over and over, 60 times a second, forever.
Each run of \texttt{draw} is called a \textit{frame}
Sketches usually begin by using one-time function that configure the sketch, like \texttt{size}.
In order to avoid running these every frame (which would slow down your sketch considerably), you can put them inside a \texttt{setup} function, which Processing will only run once at the beginning of the sketch, before going into the draw loop.
\texttt{setup} also doesn't return anything, and also takes no arguments.

While \texttt{draw} does not have any inputs, there is a special variable Processing provides named \texttt{frameCount}, which counts how many frames there have been.
The easiest way to make your sketch move is to use \texttt{frameCount} in your sketch as an input to another function, so those inputs will change over time.

The following sketch has possibly the simplest form of animation: moving a shape across the screen.
This is achieved by setting the \textit{x}-coordinate to a multiple of \texttt{frameCount}.
\begin{lstlisting}
 void setup() {
  size(1280, 720);
  colorMode(HSB, 360, 100, 100);
  background(0, 0, 25);
  fill(8, 79, 96);
  stroke(0, 0, 100);
  strokeWeight(6);
}

void draw() {
  int hh = height / 2;
  int qh = height / 4;

  ellipse(frameCount*10, hh, qh, qh);
}
\end{lstlisting}

Huzzah, it moves!
Sadly, not for very long, since the circle runs away screen right.
We can fix this by having the circle's \textit{x}-coordinate "wrap around" the width.

This "wrapping" is done by using the \textit{modulo} operator, which is represented by the \texttt{\%} symbol.
It goes in between two numbers, just like \texttt{+} does.
The value on its left is the one you want to be wrapped, and the one on its right is what you want to wrap around.
The right number, the wrap-arounder, is the limit on how big of a value comes out of the operation.

This is similar to how clocks work.
Clocks wrap around twelve.
To figure out what time it will be in five hours if it's ten right now, you add five to ten, and then since the result (fifteen) is bigger than twelve, you wrap around by subtracting twelve, giving you three.
This process is exactly how you calculate the modulo expression
\begin{lstlisting}
(10 + 5) % 12
\end{lstlisting}
\vspace{-0.5em}

\textbf{Exercise 3.6:} in the example sketch, change the circle's \textit{x}-coordinate to wrap around \texttt{width}.

\textbf{Exercise 3.7:} get rid of the "tube effect" in the example by using \texttt{background} to clear the frame at the beginning of the \texttt{draw} function.

\textbf{Exercise 3.8:} create a totally new sketch that draws some shapes.
Experiment with using frameCount as part of the arguments to draw the shapes.

\section{Now You Know Enough to be Dangerous}

These three packets are intended to provide you with a solid grounding in the concepts and execution of writing Processing sketches.
The first covered the fundamental concepts of computer graphics: the canvas coordinate system, color representation, code terminology, and drawing primitive shapes.
The second introduced two powerful coding tools: variables, and iteration.
The third showed how to animate your sketch using functions.

The first packet only lightly covers the basics of using the Processing IDE.
For a more detailed explanation, see the reference [1] below, "Getting Started. Welcome to Processing!".

If you've been working on these packets for a while but don't feel like you're making progress in your understanding, then you might have better luck with another teacher.
Try watching Dan Shiffman's "Hello Processing" video series [2].

Once you have a solid understanding of the concepts in these packets, you will be prepared to access a wealth of further Processing instructional content on the internet, not the least of which is Processing's own reference manual and tutorials section (references [3] and [4] below).
Your only limit will be your imagination.

Now, go create with code!

Cheers,\\
Dan Lidral-Porter

\begin{thebibliography}{000}
\bibitem{cit:processing-getting-started}
  \textsc{B. Fry, C. Reas}: Getting Started. Welcome to Processing! 2010\\
  \url{http://processing.org/tutorials/gettingstarted}
\bibitem{cit:hello-processing}
  \textsc{D. Shiffman, et al.}: Hello, Processing. 2013\\
  \url{http://hello.processing.org}
\bibitem{cit:processing-ref}
  \textsc{B. Fry, C. Reas, et al.}: The Processing Language Reference. 2001-\\
  \url{http://processing.org/reference}
\bibitem{cit:processing-tutes}
  \textsc{B. Fry, C. Reas, D. Shiffman, et al.}: Processing Tutorials. 2001-\\
  \url{http://processing.org/tutorials}
\bibitem{cit:color-picker}
  \textsc{Rapid Tables}: HTML Color Picker. 2011\\
  \url{http://bit.do/colorpicker}
\end{thebibliography}

\loggingall
\end{document}
\endinput
%%
%% End of file `leaflet-manual.tex'.
