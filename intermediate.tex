%%
%% This is file `leaflet-manual.tex',
%% generated with the docstrip utility.
%%
%% The original source files were:
%%
%% leaflet.dtx  (with options: `manual')
%%
%% Copyright (C) 2003, 2004
%% Rolf Niepraschk, Rolf.Niepraschk@ptb.de
%% Hubert Gaesslein, HubertJG@open.mind.de
%%
%% This work may be distributed and/or modified under the
%% conditions of the LaTeX Project Public License, either version 1.3
%% of this license or (at your option) any later version.
%% The latest version of this license is in
%%   http://www.latex-project.org/lppl.txt
%% and version 1.3 or later is part of all distributions of LaTeX
%% version 2003/12/01 or later.
%%
%% This work has the LPPL maintenance status "author-maintained".
%%
\def\filename{leaflet-manual.tex}
\def\fileversion{v1.0d}   % change this when leaflet-manual changed, too.
\def\filedate{2012/06/04}
\def\docdate {2012/06/04} % change this when leaflet-manual changed, too.
\listfiles
\errorcontextlines=99
\documentclass[
%%notumble,
%%nofoldmark,
%%dvipdfm,
%%portrait,
%%titlepage,
%%nocombine,
%%a3paper,
%%debug,
%%nospecialtricks,
%%draft,
]{leaflet}


\renewcommand*\foldmarkrule{.3mm}
\renewcommand*\foldmarklength{5mm}

\usepackage[T1]{fontenc}
\usepackage{textcomp}
\usepackage{mathptmx}
\usepackage[scaled=0.9]{helvet}
\makeatletter
\def\ptmTeX{T\kern-.1667em\lower.5ex\hbox{E}\kern-.075emX\@}
\DeclareRobustCommand{\ptmLaTeX}{L\kern-.3em
        {\setbox0\hbox{T}%
         %\vb@xt@ % :-)
         \vbox to\ht0{\hbox{%
                            \csname S@\f@size\endcsname
                            \fontsize\sf@size\z@
                            \math@fontsfalse\selectfont
                            A}%
                      \vss}%
        }%
        \kern-.12em
        \ptmTeX}
\makeatother
\let\TeX=\ptmTeX
\let\LaTeX=\ptmLaTeX
\usepackage{shortvrb}
\MakeShortVerb{\|}
\usepackage{url}
\usepackage{graphicx}
\usepackage[dvipsnames,usenames]{color}
\usepackage{listings}
\definecolor{LIGHTGRAY}{gray}{.9}

%%%%\renewcommand{\descfont}{\normalfont}
\newcommand\Lpack[1]{\textsf{#1}}
\newcommand\Lclass[1]{\textsf{#1}}
\newcommand\Lopt[1]{\texttt{#1}}
\newcommand\Lprog[1]{\textit{#1}}

\newcommand*\defaultmarker{\textsuperscript\textasteriskcentered}

\title{Processing Power Tools\vspace{-2ex}}
\author{%
  Dan Lidral-Porter\vspace{-2ex}
}
\date{}

\CutLine*{1}
\CutLine*{6}

%AddToBackground{5}{%  Background of a small page
% \put(0,0){\textcolor{Cerulean}{\rule{\paperwidth}{\paperheight}}}}
%
%AddToBackground*{2}{% Background of a large page
% \put(\LenToUnit{.5\paperwidth},\LenToUnit{.5\paperheight}){%
%   \makebox(0,0)[c]{%
%     \resizebox{.9\paperwidth}{!}{\rotatebox{35.26}{%
%       \textsf{\textbf{\textcolor{LIGHTGRAY}{BACKGROUND}}}}}}}}

\begin{document}

\definecolor{framegrey}{rgb}{0.7,0.7,0.7}

\lstset{language=Java
       ,basicstyle=\ttfamily
       ,frame=single
       ,rulecolor=\color{framegrey}
       }

\maketitle
\thispagestyle{empty}

%%\LARGE

\section{Why Use Variables?}

The purpose of a variable is to associate a name to a value in order to make both reading and writing a sketch easier.

The sketch is easier to read because names have meaning, unlike numbers.
Consider the difference between:
\begin{lstlisting}
  size(1024, 768);
  rect(512, 0, 512, 384);
\end{lstlisting}
\vspace{-0.5em}
and
\begin{lstlisting}
  size(1024, 768);
  rect(width/2, 0, width/2, height/2);
\end{lstlisting}
The first example requires mental arithmetic to understand, while the second makes it much easier to see that the rectangle covers the top-right quadrant, since it spans half the width and half the height.

Variables make sketches easier to write because you can use a variable many times, but you only have to say what it is once.
This lets you experiment easily by just changing the one line where you say what the variable is.

For example, let's say you wanted to draw a bunch of squares in a diagonal line across the canvas.
If you did that like this:
\begin{lstlisting}
  rect( 0,  0, 10, 10);
  rect(10, 10, 10, 10);
  rect(20, 20, 10, 10);
  ...
\end{lstlisting}
then if you decide that you want the squares to be bigger, you'll have to change not only all the square's sizes (the \texttt{10}s in the last two slots of each \texttt{rect} command), but also all the positions (the first two numbers), otherwise the squares will overlap.
How tedious!

If you use a variable to define the size:
\begin{lstlisting}
  // sq_w is short for 'square width'
  int sq_w = 10;
  rect(sq_w*0, sq_w*0, sq_w, sq_w);
  rect(sq_w*1, sq_w*1, sq_w, sq_w);
  rect(sq_w*2, sq_w*2, sq_w, sq_w);
  ...
\end{lstlisting}
then all you have to do is change the \texttt{10} you gave as the value for \texttt{sq\_w}, and everything works out.
How convenient!
\vspace{-1.5em}

\section{Variables in Practice}

Hopefully I've convinced you that variables are a useful tool; now let me explain how they work.

Before you can use a variable, you must at least \textit{declare} the variable, and you'll usually \textit{define} it as well.
What distinguishes a \textit{declaration} from a \textit{definition} is whether the variable is \textit{defined} to have an initial value.

If you use a variable without declaring or defining it, when you try to run your sketch the feedback area will turn red and display a message that Processing cannot find anything with the name of your undefined variable.
The line of your sketch that uses the missing variable will be highlighted so you can find the problem easily.

A variable \textit{declaration} consists of two parts: the \textit{type} of the variable; and its name.
Here's a simple declaration:
\begin{lstlisting}
  int square_width;
\end{lstlisting}
\vspace{-0.5em}
The \texttt{int} is the \textit{type} of the variable, which defines what kind of values a variable can have.
(I'll cover variable types in the next section.)
Because it's a declaration, no initial value is defined, so \texttt{square\_width} has the default value:  \texttt{0}.

A \textit{definition} starts the same as a declaration, but continues with an equals sign and an initial value that \textit{defines} the variable's value.
For example:
\begin{lstlisting}
  int square_width = 100;
\end{lstlisting}
\vspace{-0.5em}
Here the \texttt{square\_width} variable is defined to have an initial value of \texttt{100}.

\textbf{Exercise 2.1:} try declaring a variable, and then defining it. What happens? Can you figure out how to fix it?

\begin{thebibliography}{000}
\bibitem{cit:processing-ref}
  \textsc{B. Fry, C. Reas, et. al.}: The Processing Language Reference. 2001-\\
  \url{http://processing.org/reference}
\end{thebibliography}

\loggingall
\end{document}
\endinput
%%
%% End of file `leaflet-manual.tex'.
